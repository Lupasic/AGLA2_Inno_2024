% \documentclass[aspectratio=169,notes]{beamer}
\documentclass[aspectratio=169]{beamer}
\usetheme[faculty=phil]{fibeamer}
\usepackage{polyglossia}
\DeclareRobustCommand\transp{^{\mathrm{T}}}\usepackage{bm}

\setmainlanguage{english} %% main locale instead of `english`, you
%% can typeset the presentation in either Czech or Slovak,
%% respectively.
\setotherlanguages{russian} %% The additional keys allow
%%
%%   \begin{otherlanguage}{czech}   ... \end{otherlanguage}
%%   \begin{otherlanguage}{slovak}  ... \end{otherlanguage}
%%
%% These macros specify information about the presentation
\title[AGLA2]{Analytical Geometry and Linear Algebra II, Lab 7} %% that will be typeset on the
\subtitle{Eigenvalues and Eigenvectors \\ Diagonalization of a Matrix (Спектральное разложение) \\ Fast A$^N$ calculation
         } %% title page.
\author{Oleg Bulichev}
%% These additional packages are used within the document:
\usepackage{ragged2e}  % `\justifying` text
\usepackage{booktabs}  % Tables
\usepackage{tabularx}
\usepackage{tikz}      % Diagrams
\usetikzlibrary{calc, shapes, backgrounds}
\usepackage{amsmath, amssymb}
\usepackage{url}       % `\url`s
\usepackage{listings}  % Code listings
% \usepackage{subfigure}
\usepackage{floatrow}
\usepackage{subcaption}
\usepackage{mathtools}
\usepackage{todonotes}
\usepackage{fontspec}
\usepackage{multicol}
\usepackage{pdfpages}
\usepackage{wrapfig}
\usepackage{animate}
\usepackage{booktabs}
\usepackage{multirow}

\graphicspath{{resources/}}
\frenchspacing

\setbeamertemplate{caption}[numbered]
\usetikzlibrary{graphs}

% \usepackage[backend=biber,style=ieee,autocite=footnote]{biblatex}
% \addbibresource{biblio.bib}
% \DefineBibliographyStrings{english}{%
%   bibliography = {References},}

\newcommand{\oleg}[2][] {\todo[color=red, #1] {OLEG:\\ #2}}
\newcommand{\fbckg}[1]{\usebackgroundtemplate{\includegraphics[width=\paperwidth]{#1}}}%frame background

\usepackage[framemethod=TikZ]{mdframed}
\newcommand{\dbox}[1]{
\begin{mdframed}[roundcorner=3pt, backgroundcolor=yellow, linewidth=0]
\vspace{1mm}
{#1}
\vspace{1mm}
\end{mdframed}
}

\begin{document}
\setlength{\abovedisplayskip}{0pt}
\setlength{\belowdisplayskip}{0pt}
\setlength{\abovedisplayshortskip}{0pt}
\setlength{\belowdisplayshortskip}{0pt}

\fbckg{fibeamer/figs/title_page.png}
\frame[c]{\setcounter{framenumber}{0}
    \usebeamerfont{title}%
    \usebeamercolor[fg]{title}%
    \begin{minipage}[b][6.5\baselineskip][b]{\textwidth}%
        \textcolor{black}{\raggedright\inserttitle}
    \end{minipage}
    % \vskip-1.5\baselineskip

    \usebeamerfont{subtitle}%
    \usebeamercolor[fg]{framesubtitle}%
    \begin{minipage}[b][3\baselineskip][b]{\textwidth}
        \raggedright%
        \insertsubtitle%
    \end{minipage}
    \vskip.25\baselineskip
}
%   \frame[c]{\maketitle}

\fbckg{fibeamer/figs/common.png}


\begin{frame}[t]{Where it can be used}
\framesubtitle{}
\Large
    \begin{itemize}
        \item Machine learning (transform data in more suitable form)
        \item Make some calculations easier (matrix$^{100}$ – piece of cake)
        \item Predict the behavior of linear systems (physics, biology, etc)
        \item Design the controller for a system
        \item Estimate the complexity of calculations
        \item ...
    \end{itemize}
\end{frame}

\begin{frame}[t]{Definition}
\framesubtitle{}
\Large 
In linear algebra, an \textbf{eigenvector or characteristic vector }of a linear transformation is a non-zero \textit{vector that changes by only a scalar factor when that linear transformation is applied to it}. \beamerbutton{\href{https://en.wikipedia.org/wiki/Eigenvalues_and_eigenvectors}{Wiki}}
\bigskip

$ A\mathbf{x}=\lambda\mathbf{x}$, where \\ 
$x$ -- eigenvector (should be non-zero), \\ 
$\lambda$ -- eigenvalue, \\ 
$A$ -- \textit{square} matrix. \\ 
For $n \times n$ matrix -- max amount of $\lambda$ is a number of $n$ .
\end{frame}

\note{
    Рассказать про историю появления айгенов. Как думать об их появлении.
}

\begin{frame}[t]{EigenValues concept}
    \framesubtitle{Video}
    \vspace{-0.6cm}
    \begin{figure}[H]
        \href{https://youtu.be/PFDu9oVAE-g}{
            \centering\includegraphics[height=6cm,width=1\textwidth,keepaspectratio]{eigenvideo_brown.jpg}}
        % \caption{Click on a picture for a video}
        \label{fig:eigenvideo_brown.jpg}
    \end{figure}
\end{frame}

\begin{frame}[t]{Calculation (1)}
    \framesubtitle{Classical approach (max 4x4)}
    \vspace{-0.8cm}
    \begin{columns}[T,onlytextwidth]
        \begin{column}{0.48\textwidth}
            \begin{block}{Algorithm}
            There are 2 steps:
            \begin{enumerate}
                \item Find $\lambda$ (eigenvalue) --- $det(A-\lambda I)=0$
                \begin{itemize}
                    \item $2\times2$ matrix: $det(A-\lambda I) = \lambda^2 - trace(A)\lambda + det(A) = 0$, where $trace(A)$ -- sum of diag values of $A$;
                    \item $3\times3$ matrix: $det(A-\lambda I) = \lambda^3 - trace(A)\lambda^2 - \frac{1}{2}(trace(A^2)-trace(A)^2)\lambda - det(A) = 0$ 
                \end{itemize}
                \item Find $\mathbf{x}$ for each $\lambda$ --- $(A-\lambda_i I)\mathbf{x} = 0 $
            \end{enumerate}
        \end{block}
        \end{column}
        \begin{column}{0.49\textwidth}
            \begin{example}
            Case study, $2\times 2$ matrix: $A= \begin{bmatrix}
            4 & 3\\ 
            -2 & -3 
            \end{bmatrix}$
        \begin{enumerate}
            \item $trace(A)= 4 + (-3)=1$, $det(A) = 4(-3) - 3(-2)=-6$, hence \\
            $\lambda^2 - \lambda - 6 = (\lambda-3)(\lambda+2),\ \rightarrow$ \\ $\rightarrow \lambda_1 = 3,\ \lambda_2=-2$
            \item 
            \begin{enumerate}
                \item $A-3I=\begin{bmatrix}
                1 & 3\\ 
                -2 & -6 
                \end{bmatrix};\  x_{\lambda=3} = \begin{bmatrix}
                3\\
                -1
                \end{bmatrix}$
                \item  $A+2I=\begin{bmatrix}
                    6 & 3\\ 
                    -2 & -1 
                    \end{bmatrix};\  x_{\lambda=2} = \begin{bmatrix}
                    1\\
                    -2
                    \end{bmatrix}$
            \end{enumerate}
        \end{enumerate}
    \end{example}
        \end{column}
    \end{columns}
\end{frame}

\note{Спросить почему 4 на 4 максимум (ответ -- Абель в 1826 доказал, что невозможно в общем случае найти корни выше)}

\begin{frame}[t]{Task 1}
    \framesubtitle{}
    \vspace{-0.3cm}
    \begin{columns}[T,onlytextwidth]
        \begin{column}{0.49\textwidth}
            Find the eigenvalues and eigenvectors:
    \begin{enumerate}
        \item $A=\begin{bmatrix}
        2 & 7\\ 
        7 & 2 
        \end{bmatrix}$
        \item $A = \begin{bmatrix}
        3 & -1\\ 
        1 & 3 
        \end{bmatrix}$
    \end{enumerate}
        \end{column}
        \begin{column}{0.49\textwidth}
            \uncover<2->{
        \alert{\Large Answer}
        \begin{enumerate}
            \item $\lambda_1 = -5,\ \lambda_2 = 9$ \\ $x_{\lambda=-5} = \begin{bmatrix}
                -0.5\\
                0.5
                \end{bmatrix}$, $x_{\lambda=9} = \begin{bmatrix}
                    0.5\\
                    0.5
                    \end{bmatrix}$
            \item $\lambda_1 = 3+ 1i,\ \lambda_2 = 3 - 1i$ \\ $x_{\lambda=3+ 1i} = \begin{bmatrix}
                i\\
                1
                \end{bmatrix}$, $x_{\lambda=3 - 1i} = \begin{bmatrix}
                    -i\\
                    1
                    \end{bmatrix}$
        \end{enumerate}
    }
        \end{column}
    \end{columns}
    
    
\end{frame}

\begin{frame}[t]{Calculation (2)}
\framesubtitle{Real life approach (Iterative algorithms)}
    \begin{columns}[T,onlytextwidth]
        \begin{column}{0.39\textwidth}
            \large
Due to the reason that computers appeared recently, eigenpairs weren't used frequently. 
\bigskip

Nowadays, it can be found easily by iteration method, which implemented in most programming languages.

        \end{column}
        \begin{column}{0.59\textwidth}
            \begin{figure}[H]
                \centering\includegraphics[height=4cm,width=1\textwidth,keepaspectratio]{iterative.png}
                \caption*{Eigenvector and eigenvalue iterative algorithms \beamerbutton{\href{https://en.wikipedia.org/wiki/Eigenvalue_algorithm}{Wiki}}}
                \label{fig:iterative.png}
            \end{figure}
        \end{column}
    \end{columns}
\end{frame}

\begin{frame}[t]{Eigenpair properties and features}
\framesubtitle{}
\vspace{-0.3cm}
    \Large
    \begin{itemize}
        \item $\sum \lambda = trace(A)$
        \item $det(A)=\prod_{i=1}^{n}\lambda_i $
        \item $A_{new}= A_{old} + aI,\ \rightarrow$ eigenvectors won't change, $\lambda_{new} = \lambda_{old} + a$
        \item The matrix $A$ is invertible if and only if every eigenvalue is nonzero.
        \item If matrix is triangular -- the eigenvalues are on the main diagonal
        \item If matrix is symmetric -- $\lambda$ is \textit{definitely} real
        \item If matrix is not symmetric – $\lambda$ \textit{can} contain imaginary part
        \item $Ax = \lambda x \rightarrow A^2x = Aλx \text{\ (\textit{left mult})} \rightarrow A^2x = λAx \text{($\lambda$ \textit{is const})}  = \lambda^2x$ 
    \end{itemize}
\end{frame}

\note{Рассказать с помощью 1ого задания и матрицы поворота на 90 градусов про значение i (он показывает поворот)}

\begin{frame}[t]{Diagonalization}
\framesubtitle{}
    \begin{figure}[H]
        \centering\includegraphics[height=5cm,width=1\textwidth,keepaspectratio]{Diag.png}
        % \caption{caption_name}
        \label{fig:Diag.png}
    \end{figure}
\end{frame}

\note{Показать вывод ($\lambda x = x\lambda$) и отсюда уже фигачим}

\begin{frame}[t]{Diagonalization properties}
\framesubtitle{}
    \begin{figure}[H]
        \centering\includegraphics[height=6cm,width=1\textwidth,keepaspectratio]{Diag_properties.png}
        % \caption{caption_name}
        \label{fig:Diag_properties.png}
    \end{figure}
\end{frame}

\begin{frame}[t]{Diagonalization of Symmetric matrices $S = Q \Lambda Q\transp$}
\framesubtitle{}
All symmetric matrices $S$ must have \textbf{real eigenvalues} and \textbf{orthogonal eigenvectors}.
The eigenvalues are the diagonal elements of $\Lambda$ and the eigenvectors are in $Q$. 
  
  \begin{figure}[H]
    \centering
    \includegraphics[scale=1]{sqlqtr.png}
  \end{figure}
  \begin{figure}[H]
    \centering
    \includegraphics[scale=1]{pattern4.png}
  \end{figure}

  \vspace{-0.4cm}
  \centering A matrix is broken down to a sum of rank 1 matrices.

\end{frame}

\begin{frame}[t]{Task 2}
    \framesubtitle{}
\only<1>{
    $A = \begin{bmatrix}
    2 & -1\\ 
    -1 & 2 
    \end{bmatrix}$

    \begin{itemize}
        \Large
        \item Find eigenpairs;
        \item Write down $A$ in diagonal from;
        \item Draw several vectors: one, which are parallel to an eigenvector, other -- not.
        \item Multiply chosen vectors on $A$, draw the new ones.
    \end{itemize}
}
    \only<2>{
        \alert{\Large Answer}
        \begin{figure}[H]
            \centering\includegraphics[height=5.5cm,width=1\textwidth,keepaspectratio]{2ans.jpg}
            % \caption{caption_name}
            \label{fig:2ans.jpg}
        \end{figure}
        }
\end{frame}

\begin{frame}[t]{Task 3}
    \framesubtitle{}
    \vspace{-0.5cm}
    \begin{figure}[H]
        \centering\includegraphics[height=3cm,width=1\textwidth,keepaspectratio]{3.png}
        % \caption{caption_name}
        \label{fig:3.png}
    \end{figure}
    \uncover<2->{
        \alert{\Large Answer}
        \begin{figure}[H]
            \centering\includegraphics[height=3cm,width=1\textwidth,keepaspectratio]{3ans.png}
            % \caption{caption_name}
            \label{fig:3ans.png}
        \end{figure}
    }
\end{frame}

\note{ выводятся ответы из формулы диагонализации}

\begin{frame}[t]{Task 4}
    \framesubtitle{}
    \vspace{-0.5cm}
    \begin{figure}[H]
        \centering\includegraphics[height=3cm,width=1\textwidth,keepaspectratio]{4.png}
        % \caption{caption_name}
        \label{fig:4.png}
    \end{figure}
    \uncover<2->{
        \alert{\Large Answer}
        \begin{figure}[H]
            \centering\includegraphics[height=3cm,width=1\textwidth,keepaspectratio]{4ans.png}
            % \caption{caption_name}
            \label{fig:4ans.png}
        \end{figure}
    }
\end{frame}

\begin{frame}[t]{$A^k$}
\framesubtitle{}
    \begin{figure}[H]
        \centering\includegraphics[height=6cm,width=1\textwidth,keepaspectratio]{Power.png}
        % \caption{caption_name}
        \label{fig:Power.png}
    \end{figure}
\end{frame}

\note{ Вопрос и ответ выводятся ответы из формулы диагонализации}

\begin{frame}[t]{Applications (1)}
    \framesubtitle{Fast Calculations}
            \begin{columns}[T,onlytextwidth]
                \begin{column}{0.3\textwidth}
                    \underline{\textit{Task}}: Find 50th Fibonacci value \\
\textit{\underline{Dummy approach}}: calculate it by iterative summarization. \\
\underline{\textit{Smart approach}}: use magic and diagonalization \\ 
                \end{column}
                \begin{column}{0.69\textwidth}
                    \begin{figure}[H]
                        \centering\includegraphics[height=6cm,width=1\textwidth,keepaspectratio]{fibo.jpg}
                        \caption*{ \Large \href{https://youtu.be/13r9QY6cmjc?t=2072}{Lecture 22. Diagonalization and Powers of A}}
                        \label{fig:fibo.jpg}
                    \end{figure}
                \end{column}
            \end{columns}
\end{frame}

\note{Упор на клевое представление формулы в виде матрицы}

\begin{frame}[t]{Task 5}
    \framesubtitle{}
    \vspace{-0.5cm}
    \begin{figure}[H]
        \centering\includegraphics[height=3cm,width=1\textwidth,keepaspectratio]{5.png}
        % \caption{caption_name}
        \label{fig:5.png}
    \end{figure}
    \uncover<2->{
        \alert{\Large Answer}: \large \href{https://en.wikipedia.org/wiki/Stochastic_matrix}{Markov matrix} is a prob. matrix, where a summary in each column should be 1. It has a property, that $\lambda_{max} = 1$.
        \begin{figure}[H]
            \centering\includegraphics[height=3cm,width=1\textwidth,keepaspectratio]{5ans.png}
            % \caption{caption_name}
            \label{fig:5ans.png}
        \end{figure}
    }
\end{frame}

\begin{frame}[t]{Applications (2)}
\framesubtitle{Computer Vision}
\vspace{-0.3cm}
\begin{columns}[c,onlytextwidth]
    \begin{column}{0.40\textwidth}
        \textit{\underline{Task}}: we want to know the orientation of the object \\
\textit{\underline{Needed terms}}: \href{https://en.wikipedia.org/wiki/Centroid}{Centroid}, \href{https://www.youtube.com/watch?v=AAbUfZD_09s}{Image moments} \\
\textit{\underline{Solution}}: use equivalent ellipse method. We consider an ellipse:
% \vspace{-0.6cm}
\begin{itemize}
    \item centred at the object's centroid;
    \item has same moments of inertia about centroid.
\end{itemize}

Afterwards, we find an ellipse using eigenvalues and eigenvectors
    \end{column}
    \begin{column}{0.58\textwidth}
        \begin{figure}[H]
            \centering\includegraphics[height=6cm,width=1\textwidth,keepaspectratio]{cv.png}
            \caption*{\Large\href{https://robotacademy.net.au/masterclass/feature-extraction/?lesson=695}{Feature extraction masterclass video
            }}
            \label{fig:cv.png}
        \end{figure}
    \end{column}
\end{columns}

\end{frame}

\begin{frame}[t]{Applications (2.5)}
\framesubtitle{How to visualize a point cloud as an ellipse}
\vspace{-0.5cm}
    \begin{columns}[T,onlytextwidth]
        \begin{column}{0.55\textwidth}
            \textit{\underline{Task}}: We have a matrix with points. I want to make a model, which represent it in easier manner. Also, I want to visualize it. \\
\underline{\textit{Solution}}: We can find \href{https://www.youtube.com/watch?v=0GzMcUy7ZI0&feature=youtu.be}{covariance matrix} of our point cloud (It's topic from probabilistic and statistic course) and centroid of our point cloud. \\
The matrix eigenpairs provide all info (minor and major axes length and orientation) \\ 
\underline{\textit{Application}}: Eigenvectors is a basis, so we can put all our points in this basis and work with it. More info in the next semesters.
        \end{column}
        \begin{column}{0.44\textwidth}
            \vspace{-0.7cm}
            \begin{figure}[H]
                \centering\includegraphics[height=6cm,width=1\textwidth,keepaspectratio]{ellipse_point_cloud.png}
                \caption*{\large More details in matlab code below}
                \label{fig:ellipse_point_cloud.png}
            \end{figure}
        \end{column}
    \end{columns}
\end{frame}

\usebackgroundtemplate{}
\setbeamercolor{background canvas}{bg=}
\includepdf[pages=-,fitpaper]{lab7.pdf}
\fbckg{fibeamer/figs/common.png}

\begin{frame}[t]{Applications (3)}
    \framesubtitle{Machine learning + optimization}
    \vspace{-0.5cm}
        \begin{columns}[T,onlytextwidth]
            \begin{column}{0.60\textwidth}
                \underline{\textit{Task}}: we have data, which depicted on figure. We need to \textit{find local minimum} of it. \\
\underline{\textit{Dummy approach}}: let's use gradient descent w/o preprocessing. \\
\underline{\textit{Result of dummy approach}}: It can disconvergent, or solved very slow, because of  big difference between step size in x and y direction. \\
\underline{\textit{Smart approach}}: let's firstly represent it as a \textit{circle} (\textbf{transform all data in eigenbasis}) and make gradient descent on it. In this case we have almost the same step size for x and y direction.
            \end{column}
            \begin{column}{0.35\textwidth}
                \vspace{-0.5cm}
                    \begin{minipage}{0.58\textwidth}
                        \centering\includegraphics[width=\textwidth,keepaspectratio]{data_ellipse.jpg}
                    \end{minipage}\hfill
                    \begin{minipage}{0.40\textwidth}
                        Represent data as an ellipse
                    \end{minipage}
                    \begin{minipage}{0.58\textwidth}
                        \centering\includegraphics[width=\textwidth,keepaspectratio]{unnormalized.png}
                    \end{minipage}\hfill
                    \begin{minipage}{0.40\textwidth}
                        Gradient descent Issue
                    \end{minipage}
                    \begin{minipage}{0.58\textwidth}
                        \centering\includegraphics[width=\textwidth,keepaspectratio]{normalized.png}
                    \end{minipage}\hfill
                    \begin{minipage}{0.40\textwidth}
                        Represent data as a circle
                    \end{minipage}
            \end{column}
        \end{columns}
    \end{frame}

\begin{frame}[t]{Applications (4)}
    \framesubtitle{Predict the behavior of linear systems}
    \Large
    \underline{\textit{Task}}: I have a system and want to understand, how it will works. Afterwards, I want to control it (design a controller). \\
    \underline{\textit{Solution}}:
    \begin{itemize}
        \item \href{https://youtu.be/h7nJ6ZL4Lf0}{Extimate Stability using Eigenpairs}. Looking on eigenvalues we can predict stability of our linear system;
        \item \href{https://youtu.be/x_ZkKPtgTeA}{Coupled Oscillators}. Example of Eigenvalues and Eigenvectors in the context of coupled oscillators (masses connected by springs)
    \end{itemize}
\end{frame}

\begin{frame}[t]{Task 6}
    \framesubtitle{}
    \vspace{-0.5cm}
    \begin{figure}[H]
        \centering\includegraphics[height=3cm,width=1\textwidth,keepaspectratio]{6.png}
        % \caption{caption_name}
        \label{fig:6.png}
    \end{figure}
    \uncover<2->{
        \alert{\Large Answer}
        \begin{figure}[H]
            \centering\includegraphics[height=3cm,width=1\textwidth,keepaspectratio]{6ans.png}
            % \caption{caption_name}
            \label{fig:6ans.png}
        \end{figure}
    }
\end{frame}

\begin{frame}[t]{Reference material}
    \framesubtitle{}
    \Large
    \vspace{-0.5cm}
    \begin{itemize}
        \item \href{https://www.youtube.com/watch?v=lXNXrLcoerU&list=PL49CF3715CB9EF31D&index=21}{Lecture 21, Eigenvalues and Eigenvectors}
        \item \href{https://www.youtube.com/watch?v=13r9QY6cmjc&list=PL49CF3715CB9EF31D&index=22}{Lecture 22, Diagonalization and Powers of A}
        \item \textit{"Linear Algebra and Applications", pdf pages 270--306 }\\ Eigenvalues and Eigenvectors 5.1--5.3
        \item \textit{"Introduction to Linear Algebra", pdf pages 299--329 }\\ Eigenvalues and Eigenvectors 6.1--6.2
        \item  \href{https://www.youtube.com/watch?v=29keVZGvqME&list=PLkZjai-2Jcxlg-Z1roB0pUwFU-P58tvOx&index=34}{The eigenvalue problem | Lectures 32 -- 38}\\ Video from Matrix Algebra for Engineers course
    \end{itemize}
\end{frame}

\fbckg{fibeamer/figs/last_page.png}
\frame[plain]{}

\end{document}