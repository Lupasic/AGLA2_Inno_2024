% \documentclass[aspectratio=169,notes]{beamer}
\documentclass[aspectratio=169]{beamer}
\usetheme[faculty=phil]{fibeamer}
\usepackage{polyglossia}
\setmainlanguage{english} %% main locale instead of `english`, you
%% can typeset the presentation in either Czech or Slovak,
%% respectively.
\setotherlanguages{russian} %% The additional keys allow
%%
%%   \begin{otherlanguage}{czech}   ... \end{otherlanguage}
%%   \begin{otherlanguage}{slovak}  ... \end{otherlanguage}
%%
%% These macros specify information about the presentation
\title[AGLA2]{Analytical Geometry and Linear Algebra II, Lab 12} %% that will be typeset on the
\subtitle{Similar matrices \\ Singular value decomposition - SVD \\ Left and right inverses. Pseudoinverse
         } %% title page.
\author{Oleg Bulichev}
%% These additional packages are used within the document:
\usepackage{ragged2e}  % `\justifying` text
\usepackage{booktabs}  % Tables
\usepackage{tabularx}
\usepackage{tikz}      % Diagrams
\usetikzlibrary{calc, shapes, backgrounds}
\usepackage{amsmath, amssymb}
\usepackage{url}       % `\url`s
\usepackage{listings}  % Code listings
% \usepackage{subfigure}
\usepackage{floatrow}
\usepackage{subcaption}
\usepackage{mathtools}
\usepackage{todonotes}
\usepackage{fontspec}
\usepackage{multicol}
\usepackage{pdfpages}
\usepackage{wrapfig}
\usepackage{animate}
\usepackage{booktabs}
\usepackage{multirow}

\graphicspath{{resources/}}
\frenchspacing

\setbeamertemplate{caption}[numbered]
\usetikzlibrary{graphs}

% \usepackage[backend=biber,style=ieee,autocite=footnote]{biblatex}
% \addbibresource{biblio.bib}
% \DefineBibliographyStrings{english}{%
%   bibliography = {References},}

\newcommand{\oleg}[2][] {\todo[color=red, #1] {OLEG:\\ #2}}
\newcommand{\fbckg}[1]{\usebackgroundtemplate{\includegraphics[width=\paperwidth]{#1}}}%frame background

\usepackage[framemethod=TikZ]{mdframed}
\newcommand{\dbox}[1]{
\begin{mdframed}[roundcorner=3pt, backgroundcolor=yellow, linewidth=0]
\vspace{1mm}
{#1}
\vspace{1mm}
\end{mdframed}
}

\begin{document}
\setlength{\abovedisplayskip}{0pt}
\setlength{\belowdisplayskip}{0pt}
\setlength{\abovedisplayshortskip}{0pt}
\setlength{\belowdisplayshortskip}{0pt}

\fbckg{fibeamer/figs/title_page.png}
\frame[c]{\setcounter{framenumber}{0}
    \usebeamerfont{title}%
    \usebeamercolor[fg]{title}%
    \begin{minipage}[b][6.5\baselineskip][b]{\textwidth}%
        \textcolor{black}{\raggedright\inserttitle}
    \end{minipage}
    % \vskip-1.5\baselineskip

    \usebeamerfont{subtitle}%
    \usebeamercolor[fg]{framesubtitle}%
    \begin{minipage}[b][3\baselineskip][b]{\textwidth}
        \raggedright%
        \insertsubtitle%
    \end{minipage}
    \vskip.25\baselineskip
}
%   \frame[c]{\maketitle}

\fbckg{fibeamer/figs/common.png}

% \begin{frame}[c]{How I spent last weekend}
%     \framesubtitle{}
%     \begin{figure}[H]
%         \begin{subfigure}{0.29\textwidth}
%             \centering\includegraphics[height=5.5cm,width=1\textwidth,keepaspectratio]{death_door.jpg}
%         \end{subfigure}
%         \hfill
%         \begin{subfigure}{0.29\textwidth}
%             \centering\includegraphics[height=5.5cm,width=1\textwidth,keepaspectratio]{yarik.jpg}
%         \end{subfigure}        
%         \hfill
%         \begin{subfigure}{0.29\textwidth}
%             \centering\includegraphics[height=5.5cm,width=1\textwidth,keepaspectratio]{tunic_game.jpeg}
%         \end{subfigure}
%     \end{figure}
% \end{frame}

\begin{frame}[t]{Similar Matrices}
\framesubtitle{}
    
\end{frame}

\begin{frame}[c]{The primary goal of SVD}
\framesubtitle{}
\centering\LARGE
To “X-RAY” matrix

(To understand the structure of matrix)
\end{frame}

\begin{frame}[t]{Singular Value Decomposition (SVD)}
\framesubtitle{3 ways of explanation}
    \begin{columns}[T,onlytextwidth]
        \begin{column}{0.69\textwidth}
            \begin{enumerate}
                \item Linear transformation – \href{https://youtu.be/EokL7E6o1AE}{Kutz video}
                \item Algebraic – \href{https://ocw.mit.edu/courses/mathematics/18-06-linear-algebra-spring-2010/video-lectures/lecture-29-singular-value-decomposition/}{MIT video (Strang)}, \href{https://youtu.be/EfZsEFhHcNM}{Aaron Greiner video}
                \item As a tool for DS  – \href{https://youtu.be/P5mlg91as1c}{Stanford video}, \href{https://youtu.be/yA66KsFqUAE}{Brunton video}
            \end{enumerate}

            According to Kholodov words, SVD was created for \textit{finding an inverse for any matrices}. It is needed in linear transformation related operations. Other properties were found afterwards. 
        \end{column}
        \begin{column}{0.29\textwidth}
            \vspace{-1cm}
            \begin{figure}[H]
                \centering\includegraphics[height=6cm,width=1\textwidth,keepaspectratio]{two_buttons.jpg}
                % \caption{caption_name}
                \label{fig:two_buttons.jpg}
            \end{figure}
        \end{column}
    \end{columns}
\end{frame}

\begin{frame}[t]{Singular Value Decomposition (SVD)}
\framesubtitle{How to calculate it (2 possible ways)}
    \begin{block}{First approach}
        \begin{enumerate}
            \item Find eigenpairs for $A^TA$. Result is $\Sigma$ and $V$. ($A^TA=V\Sigma^{2}V^T$)
            \item Find $U$, using $\Sigma$ and $V$ ($AV\Sigma^+ = U$)
        \end{enumerate}
    \end{block}
    \begin{block}{Second approach}
        \begin{enumerate}
            \item Find eigenpairs for $A^TA$ and $AA^T$. ($A^TA=V\Sigma^{2}V^T$) ($AA^T=U\Sigma^{2}U^T$)
            \item Put signs correctly
        \end{enumerate}
    \end{block}
\end{frame}

\begin{frame}[t]{Singular Value Decomposition (SVD)}
\framesubtitle{Obtain SVD for $A= \begin{bmatrix}
    3 & 2 & 2 \\
    2 & 3 & -2 
    \end{bmatrix}$, using second approach (\href{https://jonathan-hui.medium.com/machine-learning-singular-value-decomposition-svd-principal-component-analysis-pca-1d45e885e491}{task was taken})}
\vspace{-0.8cm}
    \begin{enumerate}
        \item Eigenpairs of $AA^T$.

        $AA^T=\begin{bmatrix}
        17 & 8\\ 
        8 & 17 
        \end{bmatrix}$. $\lambda_1 = 25$, $\lambda_2 = 9$. $x_{\lambda_1} = \begin{bmatrix}
        \frac{1}{\sqrt{2}}\\
        \frac{1}{\sqrt{2}}
        \end{bmatrix}$, $x_{\lambda_2} = \begin{bmatrix}
            \frac{1}{\sqrt{2}}\\
            -\frac{1}{\sqrt{2}}
            \end{bmatrix}$. $U = \begin{bmatrix}
                \frac{1}{\sqrt{2}} & \frac{1}{\sqrt{2}}\\ 
                \frac{1}{\sqrt{2}} &  -\frac{1}{\sqrt{2}}
            \end{bmatrix}$
        \item Eigenpairs of $A^TA$.
        
        $A^TA= \begin{bmatrix}
        13 & 12 & 2 \\
        12 & 13 & -2 \\ 
        2 & -2  & 8 
        \end{bmatrix}$. $\lambda_1 = 25$, $\lambda_2 = 9$, $\lambda_3 = 0$. $V = \begin{bmatrix}
            \frac{1}{\sqrt{2}} & \frac{1}{\sqrt{18}} & \frac{2}{3} \\
            \frac{1}{\sqrt{2}} & -\frac{1}{\sqrt{18}} & -\frac{2}{3}\\ 
            0 & \frac{4}{\sqrt{18}} & -\frac{1}{3}  
        \end{bmatrix}$
    \item Result. $A\Sigma V^T=\begin{bmatrix}
        \frac{1}{\sqrt{2}} & \frac{1}{\sqrt{2}}\\ 
        \frac{1}{\sqrt{2}} &  -\frac{1}{\sqrt{2}}
    \end{bmatrix}\begin{bmatrix}
    5 & 0 & 0 \\
    0 & 3 & 2 
    \end{bmatrix}\begin{bmatrix}
        \frac{1}{\sqrt{2}} & \frac{1}{\sqrt{2}} & 0 \\
        \frac{1}{\sqrt{18}} & -\frac{1}{\sqrt{18}} & \frac{4}{\sqrt{18}}\\ 
        \frac{2}{3} & -\frac{2}{3} & -\frac{1}{3}  
    \end{bmatrix}$
    \end{enumerate}
\end{frame}

\begin{frame}[t]{It had to be $U$}  
    \framesubtitle{Video}
    \vspace{-0.6cm}
    \begin{figure}[H]
        \href{https://youtu.be/JEYLfIVvR9I}{
            \centering\includegraphics[height=6cm,width=1\textwidth,keepaspectratio]{svd_song.jpg}}
        % \caption{Click on a picture for a video}
        \label{fig:file_name}
    \end{figure}
\end{frame}

\begin{frame}[t]{Singular Value Decomposition (SVD)}
\framesubtitle{Properties}
    TODO
\end{frame}

\begin{frame}[t]{Singular Value Decomposition (SVD)}
\framesubtitle{Where it can be used}
\Large
    \begin{itemize}
        \item Image compression (slides)
        \item Pseudo-inverse (slides)
        \item Dimensionality reduction (\href{https://yadi.sk/i/kYK7E8whUsW1rA}{code + comments in pdf})
        \item Least square (\href{https://yadi.sk/i/DKxQ2F7AlN0zww}{code + comments in pdf})
        \item Principal Component Analysis (PCA) (\href{https://youtu.be/a9jdQGybYmE}{video})
        \item Eigenfaces algorithms (\href{https://youtu.be/_lY74pXWlS8}{video})
    \end{itemize}
\end{frame}

\begin{frame}[t]{SVD Applications}
\framesubtitle{Image compression}
\begin{columns}[T,onlytextwidth]
    \begin{column}{0.39\textwidth}
        \textit{\underline{Task:}} We want to compress our image for reducing the size.

\textit{\underline{Solution:}} We can represent our picture as a matrix. 

Next step is using SVD for reducing matrix rank. 

    \end{column}
    \begin{column}{0.59\textwidth}
        
    \end{column}
\end{columns}
    
\end{frame}

\begin{frame}[t]{Pseudo-Inverse}
\framesubtitle{}
    
\end{frame}

\begin{frame}[t]{Reference material}
    \framesubtitle{}
    \large
    \begin{itemize}
        \item \href{https://www.youtube.com/watch?v=TSdXJw83kyA}{Lecture 28: Similar Matrices and Jordan Form.}
        \item \href{https://www.youtube.com/watch?v=Nx0lRBaXoz4&list=PL49CF3715CB9EF31D&index=30}{Lecture 29: Singular Value Decomposition}
        \item \href{https://www.youtube.com/watch?v=Go2aLo7ZOlU&list=PL49CF3715CB9EF31D&index=34}{Lecture 33: Left and Right Inverses; Pseudoinverse}
        \item \href{https://www.youtube.com/watch?v=rYz83XPxiZo&list=PLUl4u3cNGP63oMNUHXqIUcrkS2PivhN3k&index=8}{6. Singular Value Decomposition (SVD)}
        \item \textit{"Introduction to Linear Algebra", pdf pages 375--411 }\\  7 Singular Value Decomposition (SVD)
        \item \textit{"Linear Algebra and Applications", pdf pages 335--345 }\\ 5.6 Similarity Transformations
        \item \textit{"Linear Algebra and Applications", pdf pages 377--386 }\\ 6.3 Singular Value Decomposition
    \end{itemize}
\end{frame}

\fbckg{fibeamer/figs/last_page.png}
\frame[plain]{}

\end{document}